\subsection*{Puntos adicionales}

% TODO: Hablar de forma adicional de las principales clases que se utilizaron de Swing
% los patrones de diseño que se trataron de implmentar
\subsubsection*{Interfaz gráfica}

La clase \texttt{Box} fue clave en el acomodo de los elementos ya que permite la organización en la interfaz de estos mismos
de una manera muy fácil y por medio de agruparlos en contendores llamados cajas. Se pueden crear cajas verticales u
horizontales de tal forma que los elementos se dispongan hacia los lados o de arriba hacia abajo.

En general, para la implementación del proyecto se usaron las siguientes clases de \textit{Java Swing}. Estas son solo 
algunas de las clases utilizadas para construir los elementos en la interfaz de usuario que se implementaron:

\begin{itemize}
    \item \texttt{JFrame}. Esta clase representa una ventana de una aplicación de escrito.
    \item \texttt{JPanel}. Esta clase usualmente se utiliza como contenedor de elementos de una vista de la aplicación, es uno 
    del los contendores más simples que contiene Java.
    \item \texttt{JScrollPane}. Parecida a un \texttt{JPanel}, esta clase representa un contenedor que si los elementos que contiene
    son más grande que su tamaño fijo, se podrá ver con ayuda de barras de desplazamiento tanto horizantal como vertical.
    \item \texttt{JButton}. Esta clase representa un botón dentro de la aplicación. 
    \item \texttt{JComboBox}. Esta clase representa un lista desplegable de opciones.
    \item \texttt{JLabel}. Esta clase representa una etiqueta dentro de la aplicación.
    \item \texttt{JTextField}. Esta clase representa una campo de texto donde el usuario puede escribir.
    \item \texttt{JCheckbox}. Esta clase representa un opción que puede estar seleccionada o no mediante un cuadro.
    \item \texttt{JTable}. Esta clase representa una tabla con datos.
\end{itemize}

Casi todas las anteriores clases, por no decir que todas, pueden manejar eventos. Los eventos son acciones cuyo origen es provocado
por el usuario. Algunos ejemplos son:

\begin{itemize}
    \item Dar clic en un botón.
    \item Cambiar el contenido de una campo de texto
    \item Seleccionar una opción de una lista
\end{itemize}

Estos eventos pueden ser manejados con métodos propios de las clases que usualmente empiezan con \texttt{add} y terminan con
\texttt{Listener}. Dichos métodos pueden o no aceptar funciones \texttt{lambda}; en caso que no lo hagan, reciben instancias de clases
como \texttt{ActionListener} o parececidas. Es importante destacar la utilización de expresiones \textit{lambda}, las cuales hicieron 
mucho más fácil la creación de \textit{listeners} para los diversos eventos de los componentes. Por ejemplo, en la forma tradicional 
---antes de la llegada de Java 8--- se tenía que hacer de la siguiente forma:

\begin{lstlisting}[language=Java]
    btn.addActionListener( new ActionListener() {
        @Override
        public void actionPerformed(ActionEvent e) {
            e.getSource();
        }
    });
\end{lstlisting}

O incluso de formas mucho más complejas. Sin embargo, ahora se puede simplificar todo este código a través de lo siguiente:

\begin{lstlisting}[language=Java]
    btn.addActionListener( e -> {
        e.getSource();
    });
\end{lstlisting}

A la expresión que se forma con la punta de flecha (\texttt{->}) se le conoce como expresión lambda y básicamente puede 
reemplazar a cualquier objeto que requiera implementar una interfaz que contenga un sólo método, todo lo que incluye dentro 
de las llaves es la implementación del método de la interfaz. Como se puede ver, la sintáxis es mucho más amigable y se parece 
mucho a la forma en la que trabajan mecanismos similares a los \textit{callback} o a la declaración de las funciones anónimas 
en lenguajes como \textbf{PHP} o \textbf{JavaScript}.

Las funciones \textit{lambda} abren las posibilidades a nuevos escenarios y tecnologías que se pueden aprender haciendo uso del 
lenguaje de programación Java; por ejemplo: La \textit{Java Stream API}, a través de la cual se puede escalar la forma de trabajar 
con el \textit{Java Collections Framework} y que ayuda a hacer filtros de información o ``mapeo'' de la misma, únicamente se puede 
explorar a través de funciones \textit{lambda}. Junto con este tipo de expresiones viene todo un nuevo paradigma de programación llamado 
programación funcional.

En el presente trabajo no se abordará más acerca de este tipo de expresiones ya que se considera que no es su objetivo, pero sí era 
importante enfatizar en que se hizo uso de ellas.

\subsubsection*{Patrones de diseño}
El patrón implememntado en nuestra solución fue el llamado \texttt{Singleton}. Dicho patrón consiste en que solo pueda existir una sola
instancia de la clase y ésta esté almacenada como atributo de la clase. Fue implementado en la clase \texttt{VentanaApp}. Los pasos para implementar el patrón fueron:

\begin{enumerate}
    \item Establecer un atributo de la clase del mismo tipo que la misma
    \item Establecer el modificador de acceso del constructor como \texttt{private}
    \item Crear un método público estático, llamado \texttt{getInstancia}, que regresa el atributo creado en el paso 1.
\end{enumerate}

\pagebreak