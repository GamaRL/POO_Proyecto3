\section*{Introducción}

En el presente trabajo se muestran los aspectos más generales acerca de la organización y solución
que se le dió al proyecto final de la materia de \textbf{Programación 
Orientada a Objetos}, en el cual se dió la posibilidad de elegir uno de cuatro 
posibles sistemas. Después de hacer un análisis en equipo, se llegó a la 
conclusión de que el proyecto que más nos llamaba la atención y para el que 
más ideas teníamos para desarrollar era el de \textbf{Administración de un 
Restaurante}.

Como requerimiento del proyecto y de la materia, la solución debería 
implementarse en el lenguaje de programación \textit{Java} y nuestra intención desde un 
principio fue escalarlo a un nivel de interfaz gráfica por lo que se decidió 
usar \textit{Java Swing} debido a que algunos miembros del equipo ya habían
tenido interacción con él y no se necesitan requerimientos para instalarlo (ya 
que es una biblioteca estándar del lenguaje).

Para este último proyecto, se pretendió realizar una simulación de una aplicación, sabemos
que como programadores debemos estar dispuestos a diversos retos, y así como en este caso decidimos
simular la administración de un hotel, hemos podido reconocer la gran importancia de conocer acerca de 
la \textbf{programación orientada a objetos}, pues con ayuda de esta se ha podido manejar el código de una manera 
eficaz y rápida, poniendo en práctica los conocimientos de poliformismo, herencia, interfaces, clases abstractas,
modificadores de acceso, entre muchos otros. 

Consideramos importante ver las diversas aplicaciones en las que podemos llevar a cabo todos nuestros 
conocimientos, debido a que sabemos que entender los conceptos teóricamente suele ser un poco más 
sencillo a realmente aplicarlos.

Con el desarrollo del proyecto, pretendemos llevar a cabo: 
\begin{itemize}
    \item El trabajo en equipo, que como bien sabemos es fundamental dentro y fuera de la vida laboral. 
    \item La capacidad de análisis al momento de resolver problemas reales. 
    \item Exponer todos los conocimientos adquiridos de la programación orientada a objetos, así como muchos otros
    que hemos podido ir obteniendo conforme avanzamos en la carrera. 
\end{itemize}

El código en general se encuentra dividido en diferentes carpetas cada una de estas nos ayudó a poder 
tener una mejor organización y debido a que anteriormente ya habíamos realizado algunos proyectos similares, de alguna 
forma consideramos que fue un poco más sencillo y rápido pensar la forma en la que debería de organizarse el 
proyecto. 

En esta pequeña introducción no se llevará a cabo la explicación de los componentes utilizados, sin embargo,
invitamos al lector a interesarse en la sección de \textbf{desarrollo} debido a que en ella se explicarán 
los componentes utilizados para el desarrollo total del proyecto a presentar. 

Dado que el proyecto seleccionado es un sistema para un restaurante, se cuenta con un inicio de sesión. Hay dos tipos de usuarios:
los meseros, y el o los administradores. Dependiendo del tipo de usuario se tendrán diferentes acciones a realizar.

Ambos podrán:

\begin{itemize}
    \item Tomar órdenes y llevar seguimientos de las suyas
    \item Finalizar órdenes generando su respectivo ticket
    \item Visuzalizar estadísticas como:
    \begin{itemize}
        \item Ver el o los meseros del meseros
        \item Los porcentajes de venta de cada platillo
    \end{itemize}
\end{itemize}

Solo el o los administradores podrán:

\begin{itemize}
    \item Crear y modificar perfiles con caracter de mesero
    \item Ver las ventas de meseros y platillos
\end{itemize}

% TODO: Hablar un poco más del proyecto desarrollado

\pagebreak
