\subsection*{Implementación}

% TODO: Descripción de las pricipales clases utilizadas, manejo de archivos,
% herencia, polimorfismo hablar de los archivos de objetos (mejoras sobre
% archivos de usuarios)

Para comenzar con la implementación del problema, se decidieron clasificar las
diversas clases del proyecto en las siguientes clasificaciones:

\begin{itemize}
  \item Las clases que se encargan de almacenar únicamente la información de 
  un objeto en específico (modelos).
  \item Las clases especiales para obtener/almacenar la información contenida 
  en la memoria secundaria (repositorios).
  \item Las clases que se encargan de configurar el proyecto para que pueda 
  comenzar a trabajar (configuraciones).
  \item Clases auxiliares que ayudaron a realizar tareas adicionales.
  \item Las clases encargadas de manejar la interfaz gráfica, de las cuales no 
  se entrará en tanto detalle para efectos de este escrito.
\end{itemize}

\subsubsection*{Modelos de datos}

Estas clases se encuentran definidas dentro del paquete \texttt{modelos}, en 
el cual se encuentran declarados todos y cada uno de los componentes que 
interactúan dentro de la apliación. Se comenzará hablando acerca de los 
usuarios, que fue una de las partes más importantes del sistema desarrollado; 
además de que fue las primero en implementarse. Realmente su funcionamiento
es bastante simple:

\paragraph{Clase \texttt{Usuario}, \texttt{Mesero} y \texttt{Administrador}.} 
Se tratan de las clases a través de las cuáles se implementó el inicio de 
sesión en el programa. Todas se encuentran definidas en el sub-paquete 
\texttt{modelos.usuarios}.

\begin{itemize}
  \item \textbf{\texttt{Usuario}}: Se trata de una \textit{clase abstracta} 
  que cuenta con los siguientes atributos: \texttt{id} (identificador único 
    asociado a una instancia del objeto), \texttt{nombre} (el nombre del  
    usuario), \texttt{fechaNacimiento} (la fecha de nacimiento del usuario), 
    \texttt{sexo}, \texttt{telefono}, \texttt{usuario} (nombre de usuario con 
    el que accederá a la aplicación), \texttt{password} (la contraseña con la 
    que el usuario podrá acceder al sistema).

  Cada uno de estos atributos se definió con el modificador de acceso 
  \texttt{private} con la finalidad de aplica los conceptos de 
    \textbf{encapsulamiento} y \textbf{abstracción} ---pilares fundamentales 
    de la Programación Orientada a Objetos---. Es por ello, que se definieron 
    diversos métodos de acceso para cada uno de los atributos 
    ---\textbf{\textit{setters}} y \textbf{\textit{getters}}---. Junto con dos 
    métodos adicionales de suma importancia: Por un lado,  
    \texttt{validarCredenciales}, el cual se encarga de validar si un usuario  
    y una contraseña dadas pertenecen al usuario; y por otro lado, 
    \texttt{esAdmin}, el cual se encarga de indicar si un usuario dado tiene 
    permisos de administrador o no \footnote{Esto se logra a con ayuda de la 
    clase \texttt{Administrador}}, que se mencionará más adelante.

  \item \textbf{\texttt{Mesero}}: Se trata de una clase que hereda de la clase 
    \texttt{Usuario}, únicamente define el constructor de la clase y hace uso 
    de la instrucción \texttt{super}. En esta clase se representa a los 
    meseros ``normales'' que se encuentran en un restaurante.

  \item \textbf{\texttt{Administrador}}: Se trata de una clase que hereda de la clase 
    \texttt{Usuario} ---de la misma forma que \texttt{Mesero}---, únicamente 
    define el constructor de la clase y hace uso de la instrucción 
    \texttt{super}. En esta clase se representa a los administradores del 
    restaurante ---que también tienen toda las opciones de un mesero 
    normal---.
\end{itemize}

Es importante mencionar que para la finalidad del proyecto no se requirió 
agregar más atributos o métodos a las clases \texttt{Mesero} o 
\texttt{Administrador} ya que se pude aprovechar el hecho de que cada una 
define una ``rama'' en el árbol de herencia totalmente diferente una de la 
otra. Es por ello que sólo sirven ---para efectos de nuestro programa--- como 
simples etiquetas. Pero gracias al operador \texttt{\textbf{instanceof}} se 
puede saber si un \texttt{Usuario} es administrador o no. Este comportamiento 
es el que se define en el método \texttt{esAdmin}, el cual se menciona en la 
clase \texttt{Usuario}.

\paragraph{Clase \texttt{Restaurante}.} En esta clase se definió una 
abstracción para representar un restaurante con las características 
principales para nuestro programa (\textbf{abstracción} de la realidad). Para 
ello, se analizaron los diversos requerimientos del proyecto y se llegó a la 
conclusión de que se requerían los siguentes atributos \footnote{Es importante 
mencionar que para llegar a este resultado, la versión original de la clase se
debió modificar de poco en poco para considerar nuevos requerimientos}. La 
implementación tiene lo siguiente:

\begin{itemize}
  \item \textbf{Atributos}
  \begin{itemize}
    \item \texttt{nombre}: Una cadena de texto con el nombre del restaurante.
    \item  \texttt{mesas}: Una lista de objetos de tipo \texttt{Mesa} a través 
      de la cual se almacena el estado de cada una de las mesas del resturante.
    \item \texttt{platillos}: Un objeto de tipo \texttt{Set} que se encarga de
      almacenar todos y cada uno de los platillos (objetos de la clase 
      \texttt{Platillo}) que tiene el restaurante.
    \item \texttt{tickets}: Una lista de objetos de la clase \texttt{Ticket} en 
      la que se almacenan todos los tickets que se han generado en el 
      restaurante.
  \end{itemize}
\end{itemize}

Para el caso específico de esta clase, se establecieron diversos métodos de 
acceso de consulta para cada uno de los atributos. No se implementaron métodos 
de acceso de modificación ya que no se requiere de la funcionalidad ``editar 
información del restaurante''. Se implementó el método \texttt{agregarTicket},
a través del cual se agrega un ticket a la lista de tickets del restaurante.  
Es importante mencionar que no se agregó directamente un método para ``agregar 
platillos'' ya que como tal, en el requerimiento del programa no se considera 
la funcionalidad de modificar platillos en el restaurante.

Es importante ver que todo lo que necesita un restaurante para funcionar está 
contenido en sí mismo, ya que se aplicó composición de objetos. Entonces, al 
almacenar únicamente un objeto \texttt{Restaurante}, automáticamente se 
almacenarán todos los objetos \texttt{Platillo} y \texttt{Mesa} que contiene.

El diseño de la clase \texttt{Restaurante} se llevó a cabo de tal forma que 
se pudieran hacer uso de los \textbf{archivos de objetos}, en los cuales se 
puede almacenar de manera sencilla un sólo objeto, lo cual nos convenía por 
lo expuesto anteriormente. Entonces, fue importante marcar la clase 
\texttt{Restaurante} como \texttt{Serializable} haciendo que implementara 
dicha interfaz.

\paragraph{\texttt{Platillo}. } Esta clase es la encargada de representar y 
almacenar la información referente los productos que se venden en el 
restaurante. A través de esta clase, se puede representar a platillos como:
\textit{Arroz}, \textit{Sopa de Verduras}, \textit{Enchiladas} o incluso algo 
como un \textit{Agua de Frutas} o un \textit{Refresco}. La clase en sí tiene 
los siguientes atributos:

\begin{itemize}
  \item \texttt{id}: Representa un identificador único que identificará al 
    platillo. Este atributo tiene el tipo de dato \texttt{java.util.UUID}, que 
    ya incluye toda la lógica para generar valores únicos a través del método 
    \texttt{randomUUID}.
  \item \texttt{nombre}: El nombre con el que se identificará al platillo.
  \item \texttt{precio}: Un número que indica el precio (en pesos mexicanos) 
    del platillo.
  \item \texttt{descripcion}: Un mensaje breve que indica alguna característica 
    del platillo (horario de disponibilidad, si está picoso, etc.).
\end{itemize}

con ayuda de los cuales es posible implementar las diversas funcionalidades que 
se solicitan dentro del proyecto.

Nuevamente, se vuelve indispensable tener en cuenta que como no se solicitó la 
funcionalidad de ``modificar platillos'', entoces se decidió no implementar los 
métodos de acceso de modificación para los objetos de esta clase. Los únicos 
métodos de acceso que se implementaron fueron los de consulta ---mejor conocidos
como \textit{\textbf{getters}}---.

Para efectos del programa, fue necesario sobre-excribir el método \texttt{toString} 
a través del cual se puede convertir el objeto a una cadena (\texttt{String}) 
legible para una persona. Lo que se hizo fue retornar únicamente el nombre del 
platillo en cuestión.

\paragraph{\texttt{Mesa}.} Se trata de la abstracción que se tiene de una mesa en 
un restaurante. Los atributos que define son:
\begin{itemize}
  \item \texttt{numMesa}: El número asociado con la mesa en el restaurante.
  \item  \texttt{ocupada}: Un \textit{booleano} que indica si una mesa está ocupada
    o no.
  \item \texttt{orden}: Un objeto de la clase \texttt{Orden}, a través del cual se 
    almacena la información relacionada con la orden de la mesa.
\end{itemize}

Nuevamente, se definieron los \textit{\textbf{getters}} para los atributos de la 
clase; sin embargo, también se definieron los siguientes métodos adicionales:
\begin{itemize}
  \item \texttt{setOrden}: Se trata del método de acceso de modificación para el 
    atributo \texttt{orden}.
  \item \texttt{borrarOrden}: Hace que la orden sea nula.
  \item \texttt{ocupar} y \texttt{desocupar}: Ambos conforman los métodos de acceso 
    de modificación para el atributo \texttt{ocupada}.
\end{itemize}

\paragraph{\texttt{Orden}.} Se trata de la clase que se encarga de abstraer la orden 
de una de las mesas del restaurante. En ella, se declaran los siguientes atributos:
\begin{itemize}
  \item \texttt{id}: Un objeto de la clase \texttt{java.util.UUID}, a través del cual
    se identifica a la orden de forma única.
  \item \texttt{servidor}: Un objeto que herede de la clase \texttt{Usuario} que 
    representa a la persona que está atendiendo la orden/mesa.
  \item \texttt{platillos}: Se trata de una tabla hash ---concepto abordado en la materia
    de \textit{Estructura de Datos y Algoritmos II}--- en el cual se almacenan pares de 
    valores de la forma \texttt{Platillo, Integer} para saber cuántos platillos se 
    solicitan de cada tipo. En la clase, se declara como un objeto de la clase 
    \texttt{Map}, pero a nivel de implementación se usa un \texttt{LinkedHashMap}, el cual 
    ayuda a conservar el orden de las claves (que se ve reflejado en el orden en el que
    aparecen los platillos).
  \item \texttt{mesa}: Almacena el número de la mesa con la que está asociada la orden.
\end{itemize}

Nuevamente, se definieron los \textit{\textbf{getters}} para los atributos de la clase 
más una serie de métodos adicionales en para el funcionamiento de la misma. Se comenzó por 
\texttt{calcularSubTotal}, el cual suma el precio de todos los platillos que se encuentran 
registrados en la orden. Posteriormente, se continuó con \texttt{agregarPlatillo}, el cual
agrega un platillo al \texttt{Map} que asocia los platillos con la cantidad de veces que 
se pidió. Se continuó con el método \texttt{setNumPlatillo}, el cual modifica la cantidad 
de veces que se solicitó un platillo; y por último, se implementó el método 
\texttt{eliminarPlatillo}, a través del cual se puede eliminar un platillo de la orden.

\paragraph{\texttt{Ticket}.} Se trata de la clase que se encarga de abstraer el 
ticket que se genera después de cerrar una orden. En ella se encuentran los siguientes
atributos:
\begin{itemize}
  \item \texttt{iva}: el monto que se le asigna por el costo de IVA a cuando un cliente realiza 
  una orden. Es importante mencionar que este atributo se definió como \texttt{static} para que
  se comparta a través de toda la clase y como \texttt{final} para que su valor no pueda ser 
  modificado.
  \item \texttt{mesa}: es un entero, el cual nos proporciona el número de mesa de la orden para poder 
  tener información sobre en qué mesa fue atendido cierto cliente. 
  \item \texttt{orden}: un objeto del tipo \texttt{Orden}, la cual contiene los platillos 
  que se tomaron en la mesa ocupada. 
  \item \texttt{subtotal}: contiene el monto (en pesos) de la cantidad a pagar por los platillos 
  pedidos. 
  \item \texttt{total}: el monto total a pagar, inlcuye el monto a pagar por los platillos pedidos junto con la 
  propina y el IVA. 
  \item \texttt{propina}: se refiere al monto (en pesos) que el cliente decide otorgar como propina.
  \item \texttt{esPagoConEfectivo}: de tipo \texttt{bolean} que determina si el cliente ha pagado en 
  efectivo o con tarjeta. 
  \item \texttt{fechaHora}: objetos de la clase \texttt{\textbf{LocalDateTime}} que nos permite 
  obtener la fecha y hora actual, al momento de utilizar el programa. 
\end{itemize}

\pagebreak

\subsubsection*{Repositorios}

Estas clases se encargan de hacer consultas directamente a los archivos almacenados en la memoria 
secundaria \footnote{Esta idea se tomó de \textbf{Spring Boot}, en donde las consultas a la base de datos se 
hace a través de una clase ``intermediaria'' llamada \textbf{repositorio}}. Tal como ya se mencionó 
en la sección anterior, las clases se marcaron como \texttt{Serializable} con la finalidad de poder 
almacenarlas en archivos de objetos. La forma en la que se usa la composición en la clase 
\texttt{Restaurante} nos permitió realizar de forma muy sencilla el guardado y recuperación de la 
información. Sólo requerimos tres clases ``repositorios'', ambas localizadas en el paquete 
\texttt{repositorios}, más una adicional para almacenar valores constantes para ambos repositorios.

\paragraph{\texttt{Repositorio}.} Esta clase únicamente define las rutas a los archivos de usuarios
y del restaurante. Se declaran a través de las variables ---o en este caso, ``constantes de clase'' ya 
que se definieron como \texttt{final}--- de clase \texttt{ARCHIVO\_RESTAURANTE} y 
\texttt{ARCHIVO\_USUARIOS}, junto con la variable \texttt{ruta}, la cual almacena la ruta absoluta 
al directorio de archivos y la cual se puede acceder a través del método \texttt{getRuta}.

\paragraph{\texttt{RepositorioUsuarios}.} Esta clase se encarga de hacer consulas al archivos de
usuarios, en donde se guarda la lista de usuarios que pueden entrar al sistema. Es importante mencionar
nuevamente que se hace uso de un archivo de objetos. Se tienen dos métodos importantes:

\begin{itemize}
  \item \texttt{getUsuarios}: Lee el archivo relacionado con \texttt{ARCHIVO\_USUARIOS} con la finalidad
    de recuperar la lista que almacena todos los usuarios del programa. Para ello, se abre un flujo de 
    lectura de información que fluye desde el archivo hacia el programa y a través de un ciclo 
    \texttt{do-while} se va llenando una estructura de tipo \texttt{LinkedList} con cada uno de los 
    usuarios. Es importante mencionar que se debe hacer un manejo adecuado de las excepciones: Se detecta 
    el final del archivo cuando se arroja una excepción del tipo \texttt{EOFException}, por lo que se puede 
    generar un bloque \texttt{catch} vacío y posteriormente hacer una captura de la excepción más general
    \texttt{Exception}. Además, es importante mencionar que se hizo uso de un bloque 
    \texttt{try-with-resources} para leer el archivo y este nos ayuda a cerrar automáticamente el flujo de
    datos, aún cuando sucedan errores, lo cual es sumamente conveniente.
  \item \texttt{updateUsuario}: Este métoodo recibe la lista de usaurios a ``reescribir'' y un usuario a 
    actualizar. Ahora, se abre un flujo de escritura ---que va del programa a un archivo---. Para este caso, 
    simplemente se itera sobre la lista de elementos y se detecta cuál es el elemento a actualizar con la 
    finalidad de reescribirlo con la nueva información que se proporcionó.
\end{itemize}

\paragraph{\texttt{RepositorioRestaurante}.} Esta clase se encarga de gestionar la lectura/escritura del 
objeto restaurante en un archivo de objetos. Los dos métodos con los que cuenta son:

\begin{itemize}
  \item \texttt{getRestaurante}: Nuevamente, se abre un flujo de lectura ---esta vez al archivo relacionado 
    con \texttt{ARCHIVO\_RESTAURANTE} y se lee el objeto que se encuentra ahí dentro.
  \item \texttt{guardar}: Se encarga de guardar el estado del restaurante en el archivo de objetos haciendo 
    uso de un flujo de escritura.
\end{itemize}

En cuanto el manejo de las mesas y sus respectivas órdenes se utilizaron listas, conjuntos y mapas. En el
objeto del restaurante se guardan las mesas en una lista y los platillos en un conjunto. En cada mesa se
guarda una orden, y el objeto de la orden guarda un mapa de los platillos que pertenecen a ella ---donde la 
clave es el platillo y el valor es la cantidad---. Se puede agregar, modificar la cantidad o eliminar un platillo
de una orden mediante la interfaz. Una vez que se desea finalizar la orden para generar el ticket y liberar la mesa,
se genera un objeto Ticket con la orden, la mesa y el usuario que la atendió para así poder guardarlo en la lista de
objetos Ticket la cual será muy útil en las estadísticas. También se genera un archivo de texto con la información de
la orden por cada orden finalizada. Una cosa importante que resaltar es que un mesero no puede atender una mesa que
ya es atendida por otro mesero, por lo que no será posible hacerlo en la interfaz.

Respecto a la lista de los tickets, es de importancia vital para las estadísticas y ventas. Como en esta lista se guardan
todas las órdenes ya finalizadas ---que contienen los platillos consumidos, la mesa y el servidor---, de ella se puede
recuperar la información (haciendo uso del repositorio correspondiente) para analizarla y mostrarla al usuario de una manera
comprensible.

Como se mencionó en la introducción, un administrador puede también crear y modificar perfiles de usuarios. Para ello, se hace
uso del repositorio correspondiente a usuarios (\texttt{RepositorioUsuarios}), y más específico al método \texttt{updateUsuarios}
el cual recibe la lista de usuarios, el usuario nuevo o modificado y un booleano indicando si es nuevo el usuario. Si es nuevo,
lo agrega al final del archivo; en caso contrario, va comparando con los UUID de los demás usuarios hasta que coincida para
escribir el modificado en vez del original.
\pagebreak