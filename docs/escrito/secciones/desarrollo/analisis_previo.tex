\subsection*{Análisis Previo}

% TODO: Las clases principales que se utilizaron, hablar de GitHub y de manera general de Swing.
% Forma de organización del respositorio de GitHub (directorios) y hablar de los paquetes utilizados
% en Java.
Para llevar a cabo este tercer proyecto no pasamos por alto todo lo aprendido en clase ni en los anterior
dos proyectos; estos dos últimos fueron fundamentales para la implementación de las listas, conjuntos, mapas
y patrones de diseño. Hicimos una sesión en concreto para definir nuestra jerarquía de clases y bosquejo debe
cómo estaría organizado nuestra solución en Java.

También se investigó acerca de interfaces gráficas en Java con \textit{Java Swing}. Con ayuda de documentación
oficial y conocimientos previos acerca de dichas interfaces gráficas es cómo se pudo realizar nuestra visión de
una interfaz amigable y accesible para el usuario. Toda esta parte fue hecha únicamente con código y no se uso La
herramienta que ofrece algunos IDEs ,como Netbeans, para arrastrar elementos como botones e irlos acomodando gráficamente.

Como equipo, decidimos utilizar \textit{Visual Studio Code (\textbf{VS Code})} para trabajar en el proyecto en vez de un IDE
más elaborado como \textit{NetBeans} o \textit{IntellijIDEA} para evitar configuraciones tediosas en el proyecto. 
Además, no todo el equipo contaba los mismos IDE's; y se llegó a la conclusión de que es más ligero y personalizable. 
\textbf{GitHub} fue utilizado para poder colaborar comodamente y llevar un control de versiones de nuestro proyecto.