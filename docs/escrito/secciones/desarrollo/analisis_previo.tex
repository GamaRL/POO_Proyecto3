\subsection*{Análisis Previo}

% TODO: Las clases principales que se utilizaron, hablar de GitHub y de manera general de Swing.
% Forma de organización del respositorio de GitHub (directorios) y hablar de los paquetes utilizados
% en Java.
Para llevar a cabo este tercer proyecto no pasamos por alto todo lo aprendido en clase ni en los anterior
dos proyectos; estos dos últimos fueron fundamentales para la implementación de las listas, conjuntos, mapas
y patrones de diseño. Hicimos una sesión en concreto para definir nuestra jerarquía de clases y bosquejo debe
cómo estaría organizado nuestra solución en Java.

También se investigó acerca de interfaces gráficas en Java con \textt{Java Swing}. Con ayuda de doucmentación
oficial y conocimientos previos acerca de dichas interfaces gráficas es cómo se pudo realizar nuestra visión de
una interfaz amigable y accesible para el usuario. Toda esta parte fue hecha únicamente con código y no se uso La
herramienta que ofrece algunos IDEs ,como Netbeans, para arrastrar elementos como botones e irlos acomodando gráficamente.
La clase \texttt{Box} fue clave en el acomodo de los elementos ya que permite la organización en la interfaz de estos mismos
de una manera muy fácil y por medio de agruparlos en contendores llamados cajas.

Estas son solo algunas de las clases utilizadas para construir los elementos en la GUI:

\begin{itemize}
    \item JFrame. Esta clase representa una ventana de una aplicación de escrito.
    \item JPanel. Esta clase usualmente se utiliza como contenedor de elementos de una vista de la aplicación.
    \item JScrollPane. Parecida a un JPanel, esta clase representa un contenedor que si su contenido es más grande que su tamaño
    fijo, se podrá ver con ayuda de barras de desplazamiento tanto horizantal como vertical.
    \item JButton. Esta clase representa un botón dentro de la aplicación. 
    \item JComboBox. Esta clase representa un lista desplegable de opciones.
    \item JLabel. Esta clase representa una etiqueta dentro de la aplicación.
    \item JTextField. Esta clase representa una campo de texto donde el usuario puede escribir.
    \item JCheckbox. Esta clase representa un opción que puede estar seleccionada o no mediante un cuadro.
    \item JTable. Esta clase representa una tabla con datos.
  \end{itemize}

  Casi todas las anteriores clases, por no decir que todas, puede manejar eventos. Los eventos son acciones cuyo origen es provocado
  por el usuario. Algunos ejemplos son:

  \begin{itemize}
      \item Dar clic en un botón
      \item Cambiar el contenido de una campo de texto
      \item Seleccionar una opción de una lista
  \end{itemize}

  Estos eventos pueden ser manejados con métodos propios de las clases que usualmente empiezan con \texttt{add} y terminan con
  \texttt{Listener}. Dichos métodos pueden o no aceptar funciones \texttt{lambda}; en caso que no lo hagan, reciben instancias de clases
  como \texttt{ActionListener} o parececidas.

  Como equipo, decidimos utilizar \texttt{Visual Studio Code \(VS Code\)} para trabajar en el proyecto en vez de un IDE. Esto debido a no todo
  el equipo tenían los mismos IDEs; además fue porque es más ligero y personalizable. \texttt{Github} fue utilizado para poder colaborar
  comodamente y llevar un control de versiones de nuestro proyecto.