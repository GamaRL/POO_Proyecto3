\subsection*{Puntos adicionales}

% TODO: Hablar de forma adicional de las principales clases que se utilizaron de Swing
% los patrones de diseño que se trataron de implmentar
\subsubsection*{Interfaz gráfica}
La clase \texttt{Box} fue clave en el acomodo de los elementos ya que permite la organización en la interfaz de estos mismos
de una manera muy fácil y por medio de agruparlos en contendores llamados cajas.\\

Estas son solo algunas de las clases utilizadas para construir los elementos en la GUI:

\begin{itemize}
    \item JFrame. Esta clase representa una ventana de una aplicación de escrito.
    \item JPanel. Esta clase usualmente se utiliza como contenedor de elementos de una vista de la aplicación.
    \item JScrollPane. Parecida a un JPanel, esta clase representa un contenedor que si su contenido es más grande que su tamaño
    fijo, se podrá ver con ayuda de barras de desplazamiento tanto horizantal como vertical.
    \item JButton. Esta clase representa un botón dentro de la aplicación. 
    \item JComboBox. Esta clase representa un lista desplegable de opciones.
    \item JLabel. Esta clase representa una etiqueta dentro de la aplicación.
    \item JTextField. Esta clase representa una campo de texto donde el usuario puede escribir.
    \item JCheckbox. Esta clase representa un opción que puede estar seleccionada o no mediante un cuadro.
    \item JTable. Esta clase representa una tabla con datos.
  \end{itemize}

  Casi todas las anteriores clases, por no decir que todas, puede manejar eventos. Los eventos son acciones cuyo origen es provocado
  por el usuario. Algunos ejemplos son:

  \begin{itemize}
      \item Dar clic en un botón
      \item Cambiar el contenido de una campo de texto
      \item Seleccionar una opción de una lista
  \end{itemize}

  Estos eventos pueden ser manejados con métodos propios de las clases que usualmente empiezan con \texttt{add} y terminan con
  \texttt{Listener}. Dichos métodos pueden o no aceptar funciones \texttt{lambda}; en caso que no lo hagan, reciben instancias de clases
  como \texttt{ActionListener} o parececidas.

\subsubsection*{Patrones de diseño}
El patrón implememntado en nuestra solución fue el llamado \texttt{Singleton}. Dicho patrón consiste en que solo pueda existir una sola
instancia de la clase y ésta esté almacenada como atributo de la clase. Fue implementado en la clase \texttt{VentanaApp}. Los pasos para implementar el patrón fueron:
\begin{enumerate}
    \item Establecer un atributo de la clase del mismo tipo que la misma
    \item Establecer el modificador de acceso del constructor como \texttt{private}
    \item Crear un método público estático, llamado \texttt{getInstancia}, que regresa el atributo creado en el paso 1.
\end{enumerate}