\section*{Conclusiones}

\subsection*{García Lemus, Rocío}
Considero que los objetivos planteados para el desarrollo de este proyecto se cumplieron 
en su totalidad debido a que pude implementar los conceptos aprendidos durante el curso que 
involucran a la programación orientada a objetos además de reforzarlos con ayuda de mis 
compañeros de equipo. Pienso que ha sido un gran proyecto final, pues definitivamente 
se emplearon cada uno de los conceptos vistos, de forma que pude ver reflejados
todos los conceptos aplicados a una aplicación real, en particular; mi equipo decidió escalar el 
proyecto final a una aplicación real en dónde se utilicen interfaces gráficas y sus componentes 
como bien pueden ser botones o tablas, considero que realizar el proyecto simulando la creación 
de una aplicación real fue muy beneficioso, pues considero que como programadora debo tomar en 
cuenta que, si bien durante mi etapa de formación aprendo conceptos, ideas y desarrollo el 
pensamiento lógico y analítico, al momento de terminar con esta etapa se viene el campo laboral, 
y es obvio que en esta etapa no es solamente aprender conocimientos, sino más bien es aplicar 
todos los conocimientos adquiridos durante la etapa de formación, sinceramente considero que 
tengo buenos conocimientos en cuanto al paradigma orientado a objetos, simplemente falta 
practicar aún más y tal vez lograr aplicar cada uno de los conocimientos adquiridos en 
situaciones reales. 

Trabajar en equipo fue muy importante, así como todos los aspectos que involucra el trabajo 
en equipo; la coordinación, planeación, responsabilidad y empatía, creo que he podido fortalecer 
mis habilidades del trabajo en equipo y en este caso, me encantó la forma en la que pudimos 
encajar cada uno de los integrantes; todos nos apoyamos para lograr obtener un buen resultado. 

En cuanto a los objetivos particulares planteados por nosotros mismos; considero que los hemos 
logrado totalmente, pues además de incluir los aspectos vistos en clase, decidimos escalar el 
proyecto creando una interfaz gráfica, al realizar esto decidimos considerar también el diseño de la 
interfaz, por lo que en esta ocasión pienso que además de emplear conocimientos de 
programador, pudimos entrar al papel del desarrollador gráfico para lograr una buena 
experiencia de usuario. 

\newpage

\subsection*{Juárez Juárez, María José}
Considero que este proyecto fue una buena oportunidad para consolidar los conocimientos faltantes y 
poder unir todo lo aprendido a lo largo del semestre, haciéndolo algo funcional y materializarlo, de 
manera que mediante la práctica pudiéramos comprender más a fondo el uso en conjunto de todos los 
conceptos aprendidos. 
Me parece un proyecto que abarca la mayor parte de Programación orientada a 
objetos por lo que fue de mucha utilidad para que me quedarán más claros algunos conceptos. 


Considero que la propuesta que nosotros elegimos fue una de las más completas en cuanto 
a la cantidad de elementos de POO que se usaron en ella. 
En lo personal me ayudó a entender más a fondo como funciona la herencia al utilizar 
muchas clases, polimorfismo, etc.


\newpage

\subsection*{López Chong, Jorge Antonio}
Quedo satisfecho con el producto de este proyecto porque en él se implementó desde todo lo visto en clase
(como la herencia, polimorfismo, encapsulamiento, listas, conjuntos, mapas, patrones de diseño; solo por
mencionar algunos conceptos) hasta conceptos nuevos tales como la creación y manejo de la interfaz
gráfica del usuario. Cada sesión que se tuvo fue para avanzar en el proyecto, todos aportando con su granito
de arena en una lluvia de ideas para resolver la problemática que se nos presentaba. Por todo lo anterior,
concluyo que todos los objetivos planteados en este documento se pudieron llevar a cabo en su totalidad y
satisfactoriamente.

Pienso que una área de oportunidad para el proyecto realizado sería la función de imprimir el ticket en una
impresora real, pero eso significa investigar más sobre cómo llevar a cabo es funcionalidad en Java. Aunque,
la escalabilidad del proyecto es bastante grande dado que la información vital del restaurante es guardada
en un objeto de tipo \texttt{Restaurante} por lo cual fácilmente el sistema se podría extender a varios
negocios con la finalidad de poder escalar el proyecto a un servicio en la nube con una base de datos formal
en vez de solo archivos de objetos guardados de manera local. Si se llevara a cabo esta idea, hasta se podría
implementar la creación de una aplicación móvil para que se pudiera instalar en dispostivos como celulares o
tabletas. En fin, pienso que el proyecto tienen una muy buena organización y se pudiera a llevar al siguiente
nivel con el debido esfuerzo.

Me quedo con un gran sabor de boca de la materia \texttt{Programación Orientada a Objetos} dado que las clases
y el laboratorio fueron excelentes por lo que me llevo mucho aprendizaje tanto técnico como personal. Este proyecto
fue una gran culminación del curso. Espero que para los siguientes cursos (una vez presenciales) cada equipo
pueda presentar su proyecto en una exposición de proyectos del curso o algo parecido.

\newpage

\subsection*{Ríos Lira, Gamaliel}
Considero que se cumplieron los objetivos planteados al inicio de este proyecto y que con la realización del 
mismo, ahora tenemos un mejor manejo del lenguaje de programación Java. Aplicamos la mayor parte de los conocimientos 
adquiridos durante el curso y con ello se pudo comprender la aplicabilidad que esto tiene en casi cualquier sistema 
que deseemos desarrollar.

Personalmente, considero que se cumplieron casi al 100\% las especificaciones que venían por escrito para el proyecto, 
sin embargo, en algunos casos, se decidió ``modificarlas'' un poco para realizar más sencillamente la utilización de la 
aplicación y por ello, la propuesta varía un poco con respecto a lo que se solicitó. Desde mi punto de vista, considero 
que como ingenieros en computación debemos de tener siempre la capacidad de realizar ``modificaciones'' a los 
requerimientos del cliente siempre y cuando estos ayuden a un mejor desarrollo del proyecto y sean benficiosos.

Uno de los aspectos que nos costó mucho desarrollar ---y sin embargo lo tomamos como un reto--- fue la interfaz gráfica. 
Prácticamente se tuvo que aprender una API de Java completamente desde ceros; sin embargo, esto sólo fue posible gracias 
a las buenas bases adquiridas en la materia. Se logró realizar esta parte, incluso intentamos aplicar patrón \textbf{Modelo 
Vista Controlador}, pero no nos fue posible ya que no pudimos separar la lógica de las vistas de la lógica del controlador, 
y finalmente nuestra vista hace todo el papel de vista y de controlador.

El proyecto tuvo la posibilidad de aplicar hilos o más patrones de diseño, pero decidimos mantenerlo más simple debido a la 
implementación de la interfaz gráfica (que consumió la mayor parte del tiempo).

Sumado a lo anterior, es importante mencionar que a través de la plataforma \textbf{GitHub} fue posible establecer una mejor 
colaboración para trabajar en equipo y de esta forma dividir más equitativamente el trabajo y compartir de forma más segura 
los cambios que cada integrante fue realizando.

Una de las partes que no me gustó tanto de este proyecto fue la parte de la documentación. Considero que no hemos llevado 
una materia en la que sepamos qué es una bitácora o un cronograma de actividades; sin embargo, tras una breve investigación 
logramos comprender de mejor manera de qué se trata.