\section*{Conslusiones}

\subsection*{García Lemus, Rocío}

\subsection*{Juárez Juárez, María José}

\subsection*{López Chong, Jorge Antonio}
Quedo satisfecho con el producto de este proyecto porque en él se implementó desde todo lo visto en clase
(como la herencia, polimorfismo, encapsulamiento, listas, conjuntos, mapas, patrones de diseño; solo por
mencionar algunos conceptos) hasta conceptos nuevos tales como la creación y manejo de la interfaz
gráfica del usuario. Cada sesión que se tuvo fue para avanzar en el proyecto, todos aportando con su granito
de arena en una lluvia de ideas para resolver la problemática que se nos presentaba. Por todo lo anterior,
concluyo que todos los objetivos planteados en este documento se pudieron llevar a cabo en su totalidad y
satisfactoriamente.

Pienso que una área de oportunidad para el proyecto realizado sería la función de imprimir el ticket en una
impresora real, pero eso significa investigar más sobre cómo llevar a cabo es funcionalidad en Java. Aunque,
la escalabilidad del proyecto es bastante grande dado que la información vital del restaurante es guardada
en un objeto de tipo \texttt{Restaurante} por lo cual fácilmente el sistema se podría extender a varios
negocios con la finalidad de poder escalar el proyecto a un servicio en la nube con una base de datos formal
en vez de solo archivos de objetos guardados de manera local. Si se llevara a cabo esta idea, hasta se podría
implementar la creación de una aplicación móvil para que se pudiera instalar en dispostivos como celulares o
tabletas. En fin, pienso que el proyecto tienen una muy buena organización y se pudiera a llevar al siguiente
nivel con el debido esfuerzo.

Me quedo con un gran sabor de boca de la materia \texttt{Programación Orientada a Objetos} dado que las clases
y el laboratorio fueron excelentes por lo que me llevo mucho aprendizaje tanto técnico como personal. Este proyecto
fue una gran culminación del curso. Espero que para los siguientes cursos (una vez presenciales) cada equipo
pueda presentar su proyecto en una exposición de proyectos del curso o algo parecido.

\subsection*{Ríos Lira, Gamaliel}
